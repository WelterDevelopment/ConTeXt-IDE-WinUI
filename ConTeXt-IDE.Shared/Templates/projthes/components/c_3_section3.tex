\startcomponent[c_3_section3]
\environment[env_thesis]

\startsection[title=Floats]
\startsubsection[title=Tables]
There are two table mechanisms in \ConTeXt:

\startplacetable[location=here, reference=tab:tabulate, title={Tabulate (starttabulate - stoptabulate)}]
	\starttabulate[|cw(.5\columnwidth)|lw(.5\columnwidth)|][unit=0pt]
		\FL \NC {\bf Centered column} \NC {\bf Left aligned column} \NC \AR \LL
		\NC One		\NC Two   \NC \AR
		\NC Three \NC Four  \NC \AR
	\stoptabulate
\stopplacetable

and

\startplacetable[location=here, reference=tab:table, title={TABLE (bTABLE - eTABLE)}]
	\setupTABLE[r][each][align=start]
	\setupTABLE[r][first][bottomframe=on,topframe=on]
	\setupTABLE[c][first][align=center]
	\bTABLE[split=repeat,option=stretch,frame=off]
		\bTABLEhead
			\bTR \bTH Centered column \eTH \bTH Left aligned column \eTH \eTR
		\eTABLEhead
		\bTABLEbody
			\bTR \bTD One \eTD \bTD Two \eTD \eTR
			\bTR \bTD Three \eTD \bTD Four \eTD \eTR
		\eTABLEbody
	\eTABLE
\stopplacetable

Its clear that \ref[tab:tabulate] uses less lines of code and is more \LaTeX y, where with \ref[tab:table] you get full global control on how all the rows and columns of the table are formatted.

There is also a mixture available where the code gets a little cleaner:

\startplacetable[location=here, reference=tab:mix, title={TABLE (startTABLE - stopTABLE)}]
	\setupTABLE[r][each][align=start]
	\setupTABLE[r][first][style=bold,bottomframe=on,topframe=on]
	\setupTABLE[c][first][align=center]
	\startTABLE[split=repeat,option=stretch,frame=off]
		\startTABLEhead
			\NC Centered column \NC Left aligned column \NC \NR
		\stopTABLEhead
		\startTABLEbody
			\NC One		\NC Two   \NC \NR
			\NC Three \NC Four  \NC \NR
		\stopTABLEbody
	\stopTABLE
\stopplacetable

Please note, that with the last flavour, the only usable \quote{old} table commands are \type{\NC} and \type{\NR}. You cannot use other \quote{old} table commands (like \type{\FL}, \type{\VL}) because formatting is still done with \type{\setupTABLE}. 
\stopsubsection

\startsubsection[title={Figures}]
\startplacefigure[location=here, reference=fig:examplefigure, title={Example Figure}]
	\externalfigure[pictures/NyquistPlot1.png][maxwidth=\columnwidth, height=.35\columnwidth]
\stopplacefigure
This is a reference to \ref[fig:examplefigure].

\startplacefigure[title={Subfigures}, reference=fig:combinations]
\startfloatcombination[2*2]
	\startplacesubfigure[title=Test, reference=fig:subfigure]
		\externalfigure[pictures/NyquistPlot1.png][maxwidth=\columnwidth, height=.35\columnwidth]
	\stopplacesubfigure
	\startplacesubfigure[title=Test]
		\externalfigure[pictures/NyquistPlot1.png][maxwidth=\columnwidth, height=.35\columnwidth]
	\stopplacesubfigure
\stopfloatcombination
\stopplacefigure

\stopsubsection

\stopsection
\stopcomponent